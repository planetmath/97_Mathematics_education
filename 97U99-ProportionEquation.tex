\documentclass[12pt]{article}
\usepackage{pmmeta}
\pmcanonicalname{ProportionEquation}
\pmcreated{2014-02-23 21:37:10}
\pmmodified{2014-02-23 21:37:10}
\pmowner{pahio}{2872}
\pmmodifier{pahio}{2872}
\pmtitle{proportion equation}
\pmrecord{12}{36585}
\pmprivacy{1}
\pmauthor{pahio}{2872}
\pmtype{Topic}
\pmcomment{trigger rebuild}
\pmclassification{msc}{97U99}
\pmclassification{msc}{12D99}
\pmsynonym{proportion}{ProportionEquation}
%\pmkeywords{ratio}
\pmrelated{Equation}
\pmrelated{SimilarityInGeometry}
\pmrelated{GoldenRatio}
\pmrelated{ContraharmonicProportion}
\pmrelated{ConstructionOfFourthProportional}
\pmdefines{proportion}
\pmdefines{extreme members}
\pmdefines{middle members}
\pmdefines{fourth proportional}
\pmdefines{central proportional}
\pmdefines{third proportional}

\endmetadata

% this is the default PlanetMath preamble.  as your knowledge
% of TeX increases, you will probably want to edit this, but
% it should be fine as is for beginners.

% almost certainly you want these
\usepackage{amssymb}
\usepackage{amsmath}
\usepackage{amsfonts}

% used for TeXing text within eps files
%\usepackage{psfrag}
% need this for including graphics (\includegraphics)
%\usepackage{graphicx}
% for neatly defining theorems and propositions
%\usepackage{amsthm}
% making logically defined graphics
%%%\usepackage{xypic}

% there are many more packages, add them here as you need them

% define commands here
\begin{document}
The {\it proportion equation}, or usually simply 
\PMlinkescapetext{{\it proportion}}, is an equation whose both \PMlinkescapetext{sides} are \PMlinkname{ratios}{Division} of (non-zero) numbers:
\begin{align}
           \frac{a}{b} \;=\; \frac{c}{d} 
           \;\quad \mbox{or} \;\quad a:b \;=\; c:d
\end{align}
The numbers $a$, $b$, $c$, $d$ are the {\it members} of the \PMlinkescapetext{proportion}; $a$ and $d$ are the {\em extreme members} and $b$ and $c$ are the {\em middle members}.\, The number $d$ is called the {\em fourth proportional} of the numbers $a$, $b$ and $c$.

\textbf{\PMlinkescapetext{Properties of proportions}}.
\begin{itemize}
 \item The product of the extreme members of the \PMlinkescapetext{proportion} is equal to the product of the middle members.
 \item The \PMlinkescapetext{proportion (1) is equivalent to the proportion}
$$\frac{a}{c} \;=\; \frac{b}{d},$$
i.e., the middle members can be swapped.
 \item The \PMlinkescapetext{proportion (1) is equivalent to the proportion}
$$\frac{a\!+\!b}{a\!-\!b} \;=\; \frac{c\!+\!d}{c\!-\!d}$$
if the \PMlinkescapetext{divisors} do not vanish.
 \item If any three members of a \PMlinkescapetext{proportion} are known, then the fourth member may be determined (often by using the first property).
 \item If the number $b$ satisfies the proportion
\begin{align}
\frac{a}{b} \;=\; \frac{b}{c}
\end{align}
then $b$ is called the {\em central proportional} of $a$ and $c$.\, We have
$$b \;=\; \sqrt{ac},$$
i.e., the central proportional of two real numbers (of same sign) equals to their geometric mean.
\item In (2), the number $c$ is called the {\em third proportional} of $a$ and $b$.
\end{itemize}
%%%%%
%%%%%
\end{document}
