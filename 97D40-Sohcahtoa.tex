\documentclass[12pt]{article}
\usepackage{pmmeta}
\pmcanonicalname{Sohcahtoa}
\pmcreated{2013-03-22 15:59:23}
\pmmodified{2013-03-22 15:59:23}
\pmowner{Wkbj79}{1863}
\pmmodifier{Wkbj79}{1863}
\pmtitle{sohcahtoa}
\pmrecord{13}{38010}
\pmprivacy{1}
\pmauthor{Wkbj79}{1863}
\pmtype{Definition}
\pmcomment{trigger rebuild}
\pmclassification{msc}{97D40}
\pmclassification{msc}{51-01}
\pmrelated{Adjacent2}
\pmrelated{Opposite2}
\pmrelated{Hypotenuse}
\pmrelated{Trigonometry}

\endmetadata

% this is the default PlanetMath preamble.  as your knowledge
% of TeX increases, you will probably want to edit this, but
% it should be fine as is for beginners.

% almost certainly you want these
\usepackage{amssymb}
\usepackage{amsmath}
\usepackage{amsfonts}

% used for TeXing text within eps files
%\usepackage{psfrag}
% need this for including graphics (\includegraphics)
%\usepackage{graphicx}
% for neatly defining theorems and propositions
%\usepackage{amsthm}
% making logically defined graphics
%%%\usepackage{xypic}

% there are many more packages, add them here as you need them

% define commands here

\begin{document}
Let $\theta$ be an acute angle and fix a right triangle having $\theta$ as one of its angles. The trigonometric functions $\sin \theta$, $\cos \theta$, and $\tan \theta$ can be defined as:

$$\sin \theta = \frac{\text{opposite}}{\text{hypotenuse}}$$

$$\cos \theta = \frac{\text{adjacent}}{\text{hypotenuse}}$$

$$\tan \theta = \frac{\text{opposite}}{\text{adjacent}}$$

A mnemonic device for remembering the three \PMlinkescapetext{formulas} given above is {\sl sohcahtoa\/}.

The \PMlinkescapetext{word} sohcahtoa can be divided up into three portions of three letters each.  The first portion, soh, should recall to memory that $\sin \theta$ is opposite over hypotenuse; the second portion, cah, should recall to memory that $\cos \theta$ is adjacent over hypotenuse; and the third portion, toa, should recall to memory that $\tan \theta$ is opposite over adjacent.

I would like to thank Mrs. Sue Millikin for teaching me trigonometric functions using this mnemonic device.
%%%%%
%%%%%
\end{document}
