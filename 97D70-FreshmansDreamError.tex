\documentclass[12pt]{article}
\usepackage{pmmeta}
\pmcanonicalname{FreshmansDreamError}
\pmcreated{2013-03-22 16:07:23}
\pmmodified{2013-03-22 16:07:23}
\pmowner{Wkbj79}{1863}
\pmmodifier{Wkbj79}{1863}
\pmtitle{freshman's dream error}
\pmrecord{7}{38192}
\pmprivacy{1}
\pmauthor{Wkbj79}{1863}
\pmtype{Example}
\pmcomment{trigger rebuild}
\pmclassification{msc}{97D70}

\endmetadata

% this is the default PlanetMath preamble.  as your knowledge
% of TeX increases, you will probably want to edit this, but
% it should be fine as is for beginners.

% almost certainly you want these
\usepackage{amssymb}
\usepackage{amsmath}
\usepackage{amsfonts}

% used for TeXing text within eps files
%\usepackage{psfrag}
% need this for including graphics (\includegraphics)
%\usepackage{graphicx}
% for neatly defining theorems and propositions
%\usepackage{amsthm}
% making logically defined graphics
%%%\usepackage{xypic}

% there are many more packages, add them here as you need them

% define commands here

\begin{document}
The name ``freshman's dream theorem'' comes from the fact that people who are unfamiliar with mathematics commonly make the error of distributing exponents over addition and/or subtraction, typically when working in fields of characteristic zero.  An example is the equation $(x+y)^2=x^2+y^2$ for $x,y \in \mathbb{R}$.  The equation is incorrect unless $x=0$ or $y=0$.  By no means does the exponent need to be a natural number or an integer for this error to occur.  An example of this is the equation $\sqrt{x+y}=\sqrt{x}+\sqrt{y}$ for $x,y \in \mathbb{R}$ with $x \ge 0$ and $y \ge 0$.  This equation can be rewritten using the exponent $\frac{1}{2}$, and again, the equation is incorrect unless $x=0$ or $y=0$.

An easy way to explain to someone who is under the impression that exponents distribute over addition and/or subtraction is to provide a \PMlinkescapetext{simple} counterexample.  For instance, when $x=3$ and $y=4$, we have:

\begin{center}
$\begin{array}{ccccccc}
(x+y)^2 &=& (3+4)^2 &=& 7^2 &=& 49 \\
\\
x^2+y^2 &=& 3^2+4^2 &=& 9+16 &=& 25 \end{array}$
\end{center}

On the other hand, the freshman's dream theorem yields some instances in which exponents can be distributed over addition and/or subtraction.
%%%%%
%%%%%
\end{document}
