\documentclass[12pt]{article}
\usepackage{pmmeta}
\pmcanonicalname{TopicsOnCalculus}
\pmcreated{2013-03-22 15:39:33}
\pmmodified{2013-03-22 15:39:33}
\pmowner{alozano}{2414}
\pmmodifier{alozano}{2414}
\pmtitle{topics on calculus}
\pmrecord{65}{37592}
\pmprivacy{1}
\pmauthor{alozano}{2414}
\pmtype{Topic}
\pmcomment{trigger rebuild}
\pmclassification{msc}{97A80}
\pmrelated{CommonFormulasInCalculusOfDifferentialForms}
\pmrelated{NumerableSet}
\pmrelated{NonNewtonianCalculus2}
\pmdefines{Calculus}

% this is the default PlanetMath preamble.  as your knowledge
% of TeX increases, you will probably want to edit this, but
% it should be fine as is for beginners.

% almost certainly you want these
\usepackage{amssymb}
\usepackage{amsmath}
\usepackage{amsthm}
\usepackage{amsfonts}

% used for TeXing text within eps files
%\usepackage{psfrag}
% need this for including graphics (\includegraphics)
%\usepackage{graphicx}
% for neatly defining theorems and propositions
%\usepackage{amsthm}
% making logically defined graphics
%%%\usepackage{xypic}

% there are many more packages, add them here as you need them

% define commands here

\newtheorem{thm}{Theorem}
\newtheorem{defn}{Definition}
\newtheorem{prop}{Proposition}
\newtheorem{lemma}{Lemma}
\newtheorem{cor}{Corollary}

\theoremstyle{definition}
\newtheorem{exa}{Example}

% Some sets
\newcommand{\Nats}{\mathbb{N}}
\newcommand{\Ints}{\mathbb{Z}}
\newcommand{\Reals}{\mathbb{R}}
\newcommand{\Complex}{\mathbb{C}}
\newcommand{\Rats}{\mathbb{Q}}
\newcommand{\Gal}{\operatorname{Gal}}
\newcommand{\Cl}{\operatorname{Cl}}
\begin{document}
This entry is an overview of many calculus related entries which can be found here, at PlanetMath.org. By calculus we \PMlinkescapetext{mean} real analysis at the high-school level or college level, and the entries in this page should be at either level. If an entry is written at a higher level, it will be indicated with a ``GL'' tag.

\section{Expositions}
The following are books or notes on calculus:
\begin{itemize}
\item \PMlinkexternal{An Introduction to Calculus}{http://planetmath.org/?op=getobj&from=lec&id=36}.
\end{itemize}

\section{$\Reals$ : The real line}
Basic properties of the real numbers:
\begin{itemize}
\item Decimal expansion
\item Positive
\item Inverse number, Opposite number
\item (GL) \PMlinkid{Real number}{RealNumber} 
\item Topic entry on real numbers
\end{itemize}

\section{Functions of one real variable}
\begin{itemize}
\item Concept of function and real function 
\end{itemize}

\section{Computing limits}
\begin{itemize}
\item Limit of function
\item \PMlinkname{Limit of (sin {\em x})/{\em x} as {\em x} tends to 0}{LimitOfDisplaystyleFracsinXxAsXApproaches0}
\item Limit rules of functions
\item Limit examples
\item Improper limits
\item \PMlinkname{L'H\^opital's rule}{LHpitalsRule}
\item Growth of exponential function
\item Example of computing limits using Taylor expansion
\end{itemize}

\section{Continuity}
\begin{itemize}
\item Continuous and discontinuous functions
\item Continuity of natural power
\item Continuity of sine and cosine
\item \PMlinkname{Jump discontinuity example}{ExampleOfJumpDiscontinuity}
\item Intermediate value theorem 
\end{itemize} 

\section{Differentiation in one variable}
\begin{itemize}
\item Derivative
\item Sum rule
\item Power rule
\item Product rule
\item Quotient rule
\item Chain rule
\item Derivative of inverse function
\item Derivatives of sine and cosine
\item Bolzano's theorem
\item Least and greatest value of function
\item Higher order derivatives
\item Fractional Differentiation
\end{itemize}

\section{Integration of functions of one real variable}
\begin{itemize}
\item Integral sign
\item \PMlinkname{Definition of Riemann integral}{RiemannIntegral}
\item A lecture on integration by substitution
\item A lecture on integration by parts
\item A lecture on trigonometric integrals and trigonometric substitutions
\item A lecture on the partial fraction decomposition method
\end{itemize}

\section{Definite integral}

\begin{itemize}
\item Definite integral
\item Riemann integral
\item Fundamental theorem of calculus
\item Improper integral
\item List of improper integrals
\item Approximate integration: \begin{tabular}{l}
Left Hand Rule \\
Right Hand Rule \\
Midpoint Rule \\
Trapezoidal Rule \\
Simpson's Rule \end{tabular} 
\item Integral equation
\item Fractional Integration
\end{itemize}

\section{Integral Transforms}
\begin{itemize}
\item Fourier transform
\item Laplace transform 
\item Fourier-Mellin integral
\item Mellin's transform
\item Fourier sine transform
\item Fourier cosine transform
\end{itemize}

\section{Multivariable Calculus}
\subsection{Differentiation}
\begin{itemize}
\item Iterated limit
\item Jacobian matrix
\item Differentiation under the integral sign
\end{itemize}

\subsection{Integration}
\begin{itemize}
\item Stokes' theorem
\end{itemize}

\section{Differential Equations}
\begin{itemize}
\item Differential equation
\item (GL) Existence and uniqueness of solution of ordinary differential equations
\item Index of differential equations
\item Separation of variables
\item Method of integrating factors
\item (GL) Examples of solving a PDE:\, 
 a) \PMlinkname{Heat equation}{ExampleOfSolvingTheHeatEquation}, 
b) \PMlinkname{Wave equation}{SolvingTheWaveEquationByDBernoulli} 
\end{itemize}

\section{Infinite Series}

\subsection{Series of Numbers}
\begin{itemize}
\item Series
\item Topic entry on series of complex terms
\item \PMlinkexternal{Infinite Series: Tests for Convergence and Examples}{http://planetmath.org/?op=getobj&from=lec&id=37}
\item (GL?) Non-existence of universal series convergence criterion
\item \PMlinkname{Cauchy general condition for convergence}{CauchyCriterionForConvergence}
\item Geometric series, Harmonic series
\item Sum of series depends on order
\item Manipulating convergent series
\item (GL?) Multiplication of series
\item \PMlinkname{Leibniz' estimate for alternating series}{LeibnizEstimateForAlternatingSeries}
\end{itemize}

\subsection{Function Sequences and Series}
\begin{itemize}
\item Limit of function sequence
\item The limit of a uniformly convergent sequence of continuous functions is continuous
\item Sum function of series
\item Termwise differentiation
\item Weierstrass' criterion of uniform convergence
\item (GL) \PMlinkname{Fourier series}{FourierSineAndCosineSeries}
\end{itemize}


\subsection{Power Series and Taylor Series}
\begin{itemize}
\item Power series and Taylor series
\item Taylor's theorem
\item \PMlinkname{Newton's binomial series}{BinomialFormula}
\item \PMlinkid{Example of Taylor polynomials}{ExampleOfTaylorPolynomialsForSinX} for $\sin x$
\item Example of Taylor polynomials for the exponential function
\item Examples on how to find Taylor series from other known series
\item Getting Taylor series from differential equation
\end{itemize}


\section{Additional Topic}
\begin{itemize}
\item Non-Newtonian calculus
\end{itemize}
%%%%%
%%%%%
\end{document}
