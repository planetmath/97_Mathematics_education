\documentclass[12pt]{article}
\usepackage{pmmeta}
\pmcanonicalname{InequalityWithAbsoluteValues}
\pmcreated{2013-03-22 16:57:20}
\pmmodified{2013-03-22 16:57:20}
\pmowner{pahio}{2872}
\pmmodifier{pahio}{2872}
\pmtitle{inequality with absolute values}
\pmrecord{7}{39225}
\pmprivacy{1}
\pmauthor{pahio}{2872}
\pmtype{Topic}
\pmcomment{trigger rebuild}
\pmclassification{msc}{97D40}
\pmrelated{AbsoluteValue}
\pmrelated{AbsoluteValueInequalities}
\pmrelated{OrderOfSixMeans}

% this is the default PlanetMath preamble.  as your knowledge
% of TeX increases, you will probably want to edit this, but
% it should be fine as is for beginners.

% almost certainly you want these
\usepackage{amssymb}
\usepackage{amsmath}
\usepackage{amsfonts}

% used for TeXing text within eps files
%\usepackage{psfrag}
% need this for including graphics (\includegraphics)
%\usepackage{graphicx}
% for neatly defining theorems and propositions
 \usepackage{amsthm}
% making logically defined graphics
%%%\usepackage{xypic}

% there are many more packages, add them here as you need them

% define commands here

\theoremstyle{definition}
\newtheorem*{thmplain}{Theorem}

\begin{document}
\PMlinkescapeword{line}

Recalling that the \PMlinkname{absolute value}{AbsoluteValue} of a real number means on the {\em number line} (real axis) the distance of the point from the origin, we have the following three rules which remove the absolute value signs from an inequality (we use the logical symbol ``$\lor$'' for alternativeness `or').\, Note that the symbols ``$\ge$'' and ``$\le$'' may also be without the equality bar.
\begin{enumerate}
\item \;$|a| \;\ge\; b\; \quad \Leftrightarrow \quad a \,\le\, -b \;\;\lor\;\;a \,\ge\, b$
\item \;$|a| \;\le\; b\; \quad \Leftrightarrow \quad -b \;\le\; a \;\le\; b$
\item \;$|a| \;\ge\; |b|  \quad \Leftrightarrow \quad a^2 \;\ge\; b^2$
\end{enumerate}

These rules are valid for all real values of $a$ and $b$.\, For example, if one has a case 
$$|x| \;<\; -5$$
corresponding the rule 2, this inequality seems to be impossible since no absolute value is negative; but now also the result\, $-(-5) < x < -5$\, given by the rule 2 is impossible --- no real number is simultaneously greater than $+5$ and less than $-5$.

\textbf{Examples.}\; We solve some inequalities with absolute values.

a)\; $|2x\!+\!1| > 5x$\\
$2x\!+\!1 < -5x$\; or\; $2x\!+\!1 > 5x$\,\, (rule 1)\\
$7x < -1$\; or\; $-3x > -1$\\
$x < -1/7$\; or\; $x < 1/3$\\
$x < 1/3$\,\, (combined)

b)\; $8|x|+|x\!-\!2| > 6$\\
$|8x| > 6-|x\!-\!2|$\\
$8x < -6+|x\!-\!2|$\; or\; $8x > 6-|x\!-\!2|$\,\, (rule 1)\\
$|x\!-\!2| > 8x\!+\!6$\; or\; $|x\!-\!2| > 6\!-\!8x$\\
$x\!-\!2 < -8x\!-\!6$\; or\; $x\!-\!2 > 8x\!+\!6$\; or\; $x\!-\!2 < -6\!+\!8x$\; or\; $x\!-\!2 > 6\!-\!8x$\; (rule 1 twice)\\
$9x < -4$\; or\; $-7x > 8$\; or\; $-7x  < -4$\; or\; $9x > 8$\\
$x < -4/9$\; or\; $x < -8/7$\; or\; $x > 4/7$\; or\; $x > 8/9$\\
$x < -4/9$\; or\; $x > 4/7$\,\, (from the number line)

c)\; $|1\!-\!5x| \le 3$\\
$-3 \le 1\!-\!5x \le 3$\,\, (rule 2)\\
$-4 \le -5x \le 2$\,\, (subtracted 1 from all parts)\\
$4/5 \ge x \ge -2/5$\,\, (divided by $-5)$\\
$-2/5 \le x \le 4/5$\,\, (rewritten from end to begin)
%%%%%
%%%%%
\end{document}
