\documentclass[12pt]{article}
\usepackage{pmmeta}
\pmcanonicalname{DifferenceOfSquares}
\pmcreated{2013-03-22 17:45:11}
\pmmodified{2013-03-22 17:45:11}
\pmowner{pahio}{2872}
\pmmodifier{pahio}{2872}
\pmtitle{difference of squares}
\pmrecord{10}{40204}
\pmprivacy{1}
\pmauthor{pahio}{2872}
\pmtype{Topic}
\pmcomment{trigger rebuild}
\pmclassification{msc}{97D99}
\pmclassification{msc}{26C99}
\pmclassification{msc}{13A99}
\pmsynonym{conjugate rule}{DifferenceOfSquares}
\pmrelated{ConjugationMnemonic}
\pmrelated{ExampleOnSolvingAFunctionalEquation}
\pmrelated{SquareOfSum}
\pmrelated{GroupingMethodForFactorizingPolynomials}
\pmrelated{IncircleRadiusDeterminedByPythagoreanTriple}
\pmrelated{FactoringASumOrDifferenceOfTwoCubes}
\pmrelated{Polynomial}
\pmrelated{SineOfAngleOfTriangle}
\pmrelated{RepresentantsOfQuadraticRe}

% this is the default PlanetMath preamble.  as your knowledge
% of TeX increases, you will probably want to edit this, but
% it should be fine as is for beginners.

% almost certainly you want these
\usepackage{amssymb}
\usepackage{amsmath}
\usepackage{amsfonts}

% used for TeXing text within eps files
%\usepackage{psfrag}
% need this for including graphics (\includegraphics)
%\usepackage{graphicx}
% for neatly defining theorems and propositions
 \usepackage{amsthm}
% making logically defined graphics
%%%\usepackage{xypic}

% there are many more packages, add them here as you need them

% define commands here

\theoremstyle{definition}
\newtheorem*{thmplain}{Theorem}

\begin{document}
\PMlinkescapeword{terms} \PMlinkescapeword{formula}

One of the most known and used \PMlinkname{formulas}{Equation} of mathematics is the one concerning the product of sum and difference:
\begin{align}
(a+b)(a-b) = a^2-b^2
\end{align}
This form may be used for multiplying any sum of two numbers (terms) by the difference of the same numbers (terms).\\

In the form
\begin{align}
a^2-b^2 = (a+b)(a-b)
\end{align}
the formula is used for factoring binomials which are the difference of two squares.\\

(1) is sometimes called the {\em conjugate rule}, especially in articles written in Sweden (in Swedish: {\em konjugatregel}).\\

(1) is an identic equation for all numbers $a,\,b$ and, more generally, for arbitrary elements $a,\,b$ of any commutative ring.\, Conversely, it is easy to justify that if (1) is true for all elements $a,\,b$ of a ring, then the ring is commutative.\, By the way, $a\!+\!b$ and $a\!-\!b$ also commute with each other in a non-commutative ring.

		

%%%%%
%%%%%
\end{document}
